%
% ldmx_cdr.tex
% date: July 09, 2016
%
% Set the document type and the font size
\documentclass[11pt]{article}

%%%%%%%%%%%%%%%%
%   Packages   %
%%%%%%%%%%%%%%%%
\usepackage{jheppub}	
\usepackage{url}
\usepackage{graphicx}
\usepackage{amssymb}
\usepackage{bm}
\usepackage{color}
\usepackage{mathrsfs}
\usepackage{amsmath}
\usepackage{amsfonts}
%\usepackage{hepunits}
\usepackage{color}
\usepackage{tikz}
\usepackage{slashed}
\usepackage{cancel}
\usepackage{hyperref}
\usepackage[utf8]{inputenc}											
\usepackage{slashed}
\usetikzlibrary{arrows}
\usetikzlibrary{shapes}
\usetikzlibrary{trees}
\usetikzlibrary{matrix}
\usetikzlibrary{arrows} 			
\providecommand{\draft}[1]{{\color{blue}#1}}
%%%%%%%%%%%%%%%%%%%%%%%%%%%
%%%% Gordan's packages for goals section %%%%
%%%%%%%%%%%%%%%%%%%%%%%%%%%
%\usepackage{amssymb,graphicx,cancel}
%\usepackage[margin=2cm]{geometry}
\usepackage{graphicx,epsfig,psfrag,bm,amssymb}
\usepackage{dcolumn}
\usepackage{bm}
\usepackage{color}
\usepackage{mathrsfs,amsfonts,hepunits,color}
\usepackage[utf8]{inputenc}
\usepackage{epsfig,latexsym,cancel,amssymb,amsmath,mathrsfs}
\usepackage{graphicx}
\usepackage{epstopdf}
\usepackage{mciteplus}
\usepackage{latexsym}
\usepackage{amsthm}
\usepackage{amsmath}
\usepackage{amssymb}
\usepackage{hepunits}
\usepackage{hyperref}
\usepackage{bbm}
\usepackage{bm}
\usepackage{xfrac}
\usepackage{color}
\usepackage{colordvi}
\usepackage{comment}
\usepackage{dcolumn}
\usepackage{times,latexsym,graphicx,wrapfig}
\usepackage{epsfig,lineno,bm}
\usepackage{subfigure}
\usepackage{slashed}
\newcommand{\be}{\begin{eqnarray}}
\newcommand{\ee}{\end{eqnarray}}
\newcommand{\Gev}{\,\,\mathrm{GeV}}
\newcommand{\SUWeak}{\mathrm{SU}(2)_{\rm W}}
\newcommand{\Lag}{\mathcal{L}}
\newcommand{\Lagtree}{\mathcal{L}_{\rm tree}}
\newcommand{\benum}{\begin{enumerate}}
\newcommand{\eenum}{\end{enumerate}}
\newcommand{\bi}{\begin{itemize}}
\newcommand{\ei}{\end{itemize}}
\newcommand{\met}{\slashed{E_T}}
\newcommand{\ap}{{A^\prime}}
%%%%%%%%%%%%%%%%%%%%%%%
%%%%%%%%%%%%%%%%%%%%%%%
\begin{document}

%%%%%%%%%%%%%%%%
% Front Matter %
%%%%%%%%%%%%%%%%

% 
% Front matter should be moved to its own page
%
\title{Light Dark Matter eXperiment (LDMX): \\ Letter of Intent}

\author[a]{an author}
\affiliation[a]{an institution}

\abstract{The LDMX experiment proposes a high-statistics search for low-mass dark matter at the DASEL beamline using the missing momentum technique, scattering incoming electrons in a tungsten target to produce dark matter via ``dark bremsstrahlung''. This clear signature is established by individually tagging incoming beam-energy electrons and unambiguously associating them with low energy, moderate transverse-momentum recoils and establishing the absence of a forward-going photon. The primary backgrounds are traditional bremsstrahlung processes with photo-nuclear reactions occurring in the target or forward calorimeter. Therefore, the experiment requires a high-speed, granular calorimeter with MIP sensitivity to identify rare photo nuclear reactions, in addition to low mass tracking that provides high-purity tagging for incoming electrons and clean, efficient reconstruction of recoils. The LDMX concept proposes to meet these challenges by leveraging technology under development for the HL-LHC and experience from the HPS experiment.}

\maketitle

%%%%%%%%%%%%%%%%%%%%%%%%%%%%%%%%%%%%%%
%   Overview and Executive Summary   %
%%%%%%%%%%%%%%%%%%%%%%%%%%%%%%%%%%%%%%
\section{Overview and Executive Summary (Editors: Philip Schuster)}

The LDMX experiment proposes a high-statistics search for low-mass dark matter at the DASEL beamline using the missing momentum technique, scattering incoming electrons in a tungsten target to produce dark matter via ``dark bremsstrahlung''. This clear signature is established by individually tagging incoming beam-energy electrons and unambiguously associating them with low energy, moderate transverse-momentum recoils and establishing the absence of a forward-going photon. The primary backgrounds are traditional bremsstrahlung processes with photo-nuclear reactions occurring in the target or forward calorimeter. Therefore, the experiment requires a high-speed, granular calorimeter with MIP sensitivity to identify rare photo nuclear reactions, in addition to low mass tracking that provides high-purity tagging for incoming electrons and clean, efficient reconstruction of recoils. The LDMX concept proposes to meet these challenges by leveraging technology under development for the HL-LHC and experience from the HPS experiment

%%%%%%%%%%%%%%%%%%%%%
%   Science Goals   %
%%%%%%%%%%%%%%%%%%%%%
\section{Science Goals}

\subsection{Dark Matter}
 - sub-GeV dark matter
 - summary of production kinematics

\subsection{Fundamental Forces}
 - hidden photons, scalars...etc
 - summary of production kinematics 

\subsection{Nuclear Physics}
 - rare photo-nuclear reactions


%%%%%%%%%%%%%%%%%%%%%%%%
%   Detector Concept   %
%%%%%%%%%%%%%%%%%%%%%%%%
\clearpage
\section{Detector Concept}

Basic Considerations and Overview

- signal characteristics 

- possible backgrounds 

- achieving high luminosity 

- explain why we want a tagger tracker, moderate 0.1 X0 target, recoil tracker, high granularity calorimeter, hadronic veto

- summarize the overall layout of the experiment 

\subsection{Beamline, Magnet, and Vacuum Layout}

\subsection{Tagging Tracker}

\subsection{Target}

\subsection{Recoil Tracker}

\subsection{Forward Electromagnetic Calorimeter}

\subsection{Hadronic Veto System}

\subsection{Wide-Angle Calorimeter ??}

\subsection{Trigger System}

\subsection{DAQ}


%%%%%%%%%%%%%%%%%%%%%%%%%%%%%%%%%%%%%%%
%   Physics and Detector Simulation   %
%%%%%%%%%%%%%%%%%%%%%%%%%%%%%%%%%%%%%%%
\clearpage
\section{Physics and Detector Simulation}

\subsection{Simulations Overview}

\subsection{Physics Reactions}

Explain how we handle each of: 

- signal reactions

- electromagnetic reactions of electron beam

- photo-nuclear reactions

\subsection{Detector Simulation}

Summarize the simulation of each system 


%%%%%%%%%%%%%%%%%%%%%%%%%%%
%   Performance Studies   %
%%%%%%%%%%%%%%%%%%%%%%%%%%%
\clearpage

\section{Performance Studies}

Baseline luminosity:  We want to handle $4\times 10^{14}$ electrons on target 
(EOT) with incoming energy of $4$ GeV. 


\subsection{Signal Characteristics}

Please note that we need to optimize the energy selection used below. Starting
definition: 

\textbf{Tracking:} 
\begin{itemize}
    \item One incoming beam electron with $E_{beam}$ close to $4$ GeV and a well
          measured trajectory. 
    \item Quality cuts (as needed) to reduce any dangerous brem 
          (or photo-nuclear) reactions in the tagger tracker material
    \item One recoiling electron with $E\lesssim 1.2$ GeV that points back to 
          the incoming beam electron track. 
    \item An activity cut in the recoil tracker to reject photo-nuclear 
          reactions in the target
    \item An inferred ``missing momentum'' trajectory and magnitude 
\end{itemize}

\textbf{Calorimetry:}
\begin{itemize}
    \item One soft recoil shower with $E\lesssim 1.2$ GeV that is consistent 
          with recoil tracker trajectory
    \item An activity cut in the ``missing momentum'' region for the ECal
    \item An explicit veto on energetic (energy range needs to be specified)
          hadrons in both the ECal and hadron veto system 
\end{itemize}



\subsection{Tagging Tracker Performance (Editors: Tim Nelson, Omar Moreno)}

The tagging tracker must identify incoming beam-energy electrons with extremely high purity, suppressing the mis-reconstruction of any incoming low-momentum charged particles as beam energy electrons. In particular, any incoming charged particle within the recoil acceptance for signal that is reconstructed as having the beam energy in the tagging tracker is an irreducible background. The design of the tagging tracker makes the likelihood of such errors vanishingly small, with good resolution for both beam energy and off-energy incoming tracks and an exceedingly low rate of mis-reconstruction for tracks within the recoil energy acceptance.

In order for an incoming low momentum particle to fake a beam energy electron in the tagger, a number of conditions must simultaneously be met:
\begin{enumerate}
    \item The incoming particle must reach the first tagger layer or it will not intersect with any material until it hits the wall of the vacuum chamber.
    \item The particle must either scatter in each layer in order to fake a 4 GeV track or create secondaries that generate occupancy which confuses the pattern recognition in the tracker resulting in reconstruction of a fake 4 GeV track.
    \item The resulting track must have a trajectory consistent with that of a typical 4 GeV beam electron all the way through the tracker.
    \item The resulting track must have an impact point at the target consistent with a reconstructed track within the signal acceptance in the recoil tracker. 
\end{enumerate}
Using an analytic model of the tagging tracker that includes the effect of intrinsic resolutions and multiple scattering in the tracker planes, it is evident that each of these requirements places a very heavy penalty on any off-energy component in the beam.  First, incoming particles with less than approximately 500 MeV momentum will not hit the first layer of the tagger unless they are significantly off-trajectory as well.  Furthermore, even at 500 MeV, a first scatter of more than 10$^\circ$ is needed in order for the incoming particle to appear to be on the correct trajectory. It is clear then that the most challenging scenario is large contamination with incoming charged particles at the top of the momentum range for signal recoils, nominally 1.2 GeV.  Such particles have the highest likelihood of reaching the first layer of the tagger tracker without being bent away by the magnetic field and require much smaller scatters and/or track reconstruction errors to result in fake tags. In order for a 1.2 GeV particle to make a trajectory through the tracker consistent with a 4 GeV track, six successive scatters of approximately ten milliradians must occur, each equivalent to approximately 15$\sigma$ on the multiple scattering distribution. From the tails of the Moliere scattering distribution, the likelihood of each of these scatters is smaller than one per million.  Therefore, the much more likely scenario is the generation of secondaries an the material of the tagger tracker followed by misreconstruction of a fake 4 GeV track.  Since the resulting 4 GeV track must arrive at the target on the correct trajectory and beam energy electrons arrive normal to the target with a one-sigma error of 250 microradians, there is very little phase space for randomly reconstructed 4 GeV fakes to have the correct trajectory.  Finally, any falsely reconstructed 4 GeV track must have a common impact point in the target with a real track of matching momentum in the recoil tracker, which is unlikely for a falsely reconstructed tracks.

In order to more fully test these scenarios, two samples of incident electrons were simulated and reconstructed in the tagger tracker.  The first is a sample of XXX 4 GeV electrons on the nominal beam trajectory.  The second is a sample of XXX 1.2 GeV electrons on a trajectory that allows them to pass through all seven layers of the tagging tracker.  The 4 GeV sample confirms the expected resolutions, as shown in
Figures~\ref{fig:trackin_4gev_p}.  
%=======================
\begin{figure}[htp]
    \centering
    \includegraphics[width=\textwidth]{images/tagger_tracker_p_pt_4pt0_gev.png}
    \caption{\small{Reconstructed total momentum and momentum transverse to 
                    target for a sample of 4 GeV beam electrons.}}
    \label{fig:tracking_4gev_p}
\end{figure}
%=======================
%=======================
\begin{figure}[htp]
    \centering
    %    \includegraphics[width=7cm]{figures/dummy}
    \caption{\small{Reconstructed Tagger track $x$-$y$ position at the target for a
    sample of 4 GeV beam electrons.}}
    \label{fig:tracking_4gev_pos}
    %\vspace*{-5mm}
\end{figure}
%=======================

These indicate that tight requirements can be made in both the energy and trajectory at the target that rejects off-momentum particles that could be present in the incoming beam and that the tagging tracker identifies a precise impact position that can be used for tracking recoil candidates.  The 1.2 GeV sample confirms at the level of 1 part in 10$^?$ that these tracks cannot be mistaken for 4 GeV tracks, as shown in Figure~\ref{fig:tracking_1pt2gev}.  
%=======================
\begin{figure}[htp]
    \centering
    \includegraphics[width=\textwidth]{images/tracker/tagger_tracker_p_pt_1pt2_gev.pdf}
    \caption{\small{Reconstructed total momentum for a sample of 1.2 GeV beam 
                    electrons.}}
    \label{fig:tracking_1pt2gev}
\end{figure}
%=======================
In order to probe the likelihood of reconstructing fake 4 GeV tracks at higher statistics we further introduce random noise hits on all planes of the tracker at rates of $10^{-3}$, $10^{-2}$ and $10^{-1}$ in all planes of the tracker, where typical noise occupancies in similar HPS modules are roughy $10^{-4}$. Figure~\ref{fig:tracking_1pt2gev_noise} shows the resulting distribution of reconstructed tracks in energy and $p_T$ at the target. Obviously, such extreme occupancies in the tracker are atypical, and would easily be selected against with negligible impact on signal efficiency. 
%=======================
\begin{figure}[htp]
    \centering
    %    \includegraphics[width=7cm]{figures/dummy}
    \caption{\small{Reconstructed momentum vs. momentum transverse to the target for a sample of 1.2 GeV electrons in the presence of $10^{-3}$, $10^{-2}$ and $10^{-1}$ random occupancy in all sensors.} }
    \label{fig:tracking_1pt2gev_noise}
    %\vspace*{-5mm}
\end{figure}
%=======================
Furthermore, these fake tracks, not being due to an individual low-momentum track, will typically not align with a low-momentum track in the recoil tracker.  Although further study will be required to find the beam intensity limits for this tagging tracker design, we can safely conclude that it is more than capable of providing the tagging purity required for the first stage of the LDMX experiment.


\subsection{Recoil Tracker Performance, (Editors: Tim Nelson, Omar Moreno)}

The recoil tracker must have a large acceptance for recoiling electrons characteristic of signal events with good resolution for transverse momentum and impact position at the target which are critical for unambiguously associating those recoils with incoming electrons identified by the tagging tracker.  While good reconstruction efficiency for signal recoils is important, it is even more important to have good efficiency for charged tracks over the largest possible acceptance to help the calorimeter veto background events with additional charged particles in the final state.  In addition, the tracker should have sufficient momentum resolution that it can assist the ECal in identifying events where an incoming beam electron passes through the target and tracker without significant energy loss. Finally, the recoil tracker can help identify events where a bremsstrahlung photon generated after the end of the tagger tracker undergoes a photonuclear reaction in the target or tracker material.

Estimation of the impact parameter and recoil momentum transverse to the target require precise determination of the angle at the target.  For the transverse momentum, a good curvature measurement is also required to set the overall momentum scale.  At least two 3-d measurements directly downstream of the target are needed to determine the recoil angle while at least one additional bend-plane measurement is needed for curvature.  For low-momentum tracks, this third measurement can be in the first four closely-spaced layers, but for high momentum tracks that are nearly straight, hits in both of the downstream axial layers are required for good momentum resolution.  Because high-momentum signal recoils will nearly always pass through all six layers, the acceptance near the top of the energy range for signal recoils is near unity, only reduced by the single-hit efficiency in the last two layers. However, at low momentum a large number of tracks can escape detection. Therefore, in order to estimate the signal acceptance for well-measured recoils, we require that recoiling electrons leave hits in at least three of the 3-d layers, for pattern recognition and angle estimation, and at least four hits total for reasonably purity.  The tracker acceptance for signal recoils as a function of mediator mass is shown in Figure~\ref{fig:signal_plots} along with the track finding efficiency, important for eliminating events with extra charged particles in the final state.

%%%%%%%%%%%%%%
%   Figure   %
%%%%%%%%%%%%%%
\begin{figure}[htp]
    \centering
    \includegraphics[width=\textwidth]{images/tracker/signal_acceptance_eff.png}
    \caption{The recoil tracker acceptance to signal recoils as a function of 
             $A'$ mass (left) alogn with the track finding efficiency.}
    \label{fig:signal_plots}
\end{figure}
%%%%%%%%%%%%%%
%%%%%%%%%%%%%%

The ability to distinguish signal from background using the recoil transverse momentum is obviously limited by the multiple scattering in the target, where multiple scattering in a 10\% $X_0$ target results in a 4 MeV smearing in transverse momentum.  Using an analytic model of the target and recoil tracker, the material budget and single-hit resolutions of the recoil tracker have been designed so that the experimental resolution is limited by multiple scattering in the target over the entire momentum range for signal recoils. This has been verified in full simulation as shown in Figure XXX.  Furthermore, the impact parameter resolution, also shown in Figure XXX, results in a very low probability that a mis-reconstructed recoil track can be incorrectly associated with a tagged incoming electron.  

While the ECal does an excellent job distinguishing scattered full energy electrons from potential signal recoils, the recoil tracker provides significant additional leverage for this critical task. In order to estimate the contamination due mis-reconstructed full energy electrons in the momentum range for signal recoils, $p<1.2$ GeV, we examine a large sample of incident 4 GeV electrons.  Figure XXX shows that there is a low-momentum tail in the reconstructed tracks, but that these tracks are accompanied by bremsstrahlung photons in the final state that are easily vetoed.

Finally, events where a hard bremsstrahlung occurs after the tagger tracker and undergoes a photonuclear reaction in the target or recoil tracker will sometimes be observable in the recoil tracker.  In the case of photonuclear reaction in the target, large multiplicity in the first layers of the recoil tracker can result, as shown in Figure XXX.  In the case of photonuclear reaction in the silicon, a hit with very large charge is often observed, as shown in Figure XXX.  Along with the trigger pad (see Section XXX), use of this information can help reduce backgrounds due to photonuclear events that occur upstream of the calorimeters.






\subsection{Forward Electromagnetic Calorimeter (Editors: Joe Incandela)}

Owen and Joe will discuss this at the July 8 meeting. But Philip's notes include: 

- Hermiticity:  make sure to include a study of cracks or dead material in the detector simulation. Do we need to worry about this? Why? 

Large scale ``top down'' monte carlo study to demonstrate baseline performance of ECal and to justify more detailed study of specific reactions that dominate the tail of low energy deposition events. We need to quantify everything I'm about the say more carefully. Starting from $4\times 10^{14}$ EOT, the baseline tracker selections bring the event sample down to $\sim 4 \times 10^{12}$. So we're dealing with $\sim 4 \times 10^{12}$ events with a soft recoiling electron and a hard, $\sim 3$ GeV, photon. ECal events that are hadron rich occur about $\sim10^{-3}$ of the time. So now we're down to $\sim 4\times 10^{9}$ hadron rich events in the ECal. 

- most importantly, we want to understand what dominates the low energy deposition events

- we want to characterize the hadron rich events (because we know they are a potential issue)

- explain veto strategy

- at what point is the energy deposition so low that it's not possible to veto effectively? what are these events types? 

Specific ``bottom up'' studies of photo-nuclear reactions:  We know that certain event types could pose a challenge, so let's study them. The numbers shown below are with very loose kinematic selections, so they are upper bounds. They are also for $9$ GeV photons, so Philip and Natalia will need to correct them. This is a good starting point for study however: 

- $\gamma N \rightarrow (\rho,\omega,\phi)N\rightarrow \pi^+\pi^- N$ ($\lesssim 10^8$ of this event type). 

- $\gamma N \rightarrow  \mu^+\mu^- N$ ($\lesssim 2 \times 10^7$ of this event type). 

- $\gamma p \rightarrow \pi^+ n$ ($\lesssim 4\times 10^5$ with $\sim 4\times 10^3$ of these having a backscattered $\pi^+$). 

- $\gamma n \rightarrow n \bar{n} n$ ($\lesssim 4\times 10^5$ of this type). 

- $\gamma (p,n) \rightarrow K_L K_L + X$ (expect this at the $\sim 10^3$ level, but we need to check this!)

The current plan is to use a particle gun and weight the depth of origination and angle/energy distribution using data. Let Philip and Natalia know when you're ready to do this. 
{\it We need to be especially careful to include the regions of phase space where the MIPs are soft or wide/back scattering by recoiling off the nucleons or atoms. This needs a dedicated study, starting with the physics simulations group, P,N,E,G}. 





\subsection{Hadronic Veto System (Editors: Jeremy Mans, Nhan Tran, Andrew Whitbeck)}

The hadronic veto system is designed for detecting rare processes and therefore studies of its performance focus on "bottom up" vetoes of photo-nuclear reactions.  
If we consider that the rare processes described in Section~\ref{} should produce a low multiplicity of neutrons we perform studies to compute on the efficiency the hadronic veto system 
of a detecting single neutron.  
This will give us an idea of the efficiency with which we can veto any type of photo-nuclear reaction based on the hadronic system only.  

We benchmark the hadronic veto system by considering neutrons, being generated at the face of the calorimeter, with various:
\begin{itemize}
\item energies: 1.3, 1.5, 1.8, 2.0, 2.2, 2.5, 3.5~GeV (total energy)
\item number of calorimeter layers: 15, 20, 25
\item incident angles: 0, 15, 30~degrees
\end{itemize}
For each neutron phase space point, we generate $5 \times 10^4$ events.  

First we study the propagation of neutrons through the HCAL purely via {\tt GEANT} without yet considering the digitization of the scintillation signal, which was described in Section~\ref{sec:hcaldig}.
We define the kinetic energy flux for a given layer as the amount of kinetic energy passing through the front face of a given HCAL layer.  
In Fig.~\ref{fig:neturonflux}, we show the neutron kinetic energy flux as a function of HCAL layer for 3.5~GeV total energy neutrons produced at an incident angle of 0.0$^{\circ}$.
We note that the "shower max" for hadronic showers in this system is typically around layers 6/7.  
By looking at the neutron flux around a kinetic energy of 2.5~GeV as a function of layers, we can estimate the fraction of neutrons which pass directly through the calorimeter without interacting.  
We can conclude that roughly $100/50000 \sim 0.2\%$ of neutrons do not interact in the first 15 layers of the system.
This number drops to roughly $0.02\% (0.002\%)$ for a 20 (25) layer system.
This fraction is independent of incident neutron energy.

\begin{figure}[hbtp]
\begin{center}
    \includegraphics[width=0.5\textwidth]{images/hcal/NFlux_vs_Layer_e3d5_th0d0.pdf}
    \caption{Neutron kinetic energy flux through the HCAL as a function of layer for a 3.5~GeV neutron produced with an incident angle of 0.0$^{\circ}$ at the front face of the calorimeter.}
 \label{fig:neturonflux}
 \end{center}
\end{figure}

For neutrons that do interact in the HCAL, we then compute the efficiency of detecting them given a simple digitization process described in Section~\ref{sec:hcaldig}.
As a reminder, we assume that a MIP deposits, on average, 1.4~MeV of energy in the scintillator, which from CMS testbeam studies we estimate translates into 13.5 photo-electrons at the SiPM.  
Given a typical noise contribution of 2 photo-electrons in SiPMs, we define a MIP signal in a given layer as 8 photo-electrons.  
Therefore, we define a {\it MIP layer} as a scintillator layer in the HCAL in which enough energy is deposited to produce at least 8 photo-electrons (Poisson-varied).   
We define a vetoed neutron event as a neutron which has created $\geq 1$ MIP layer. 
The number of MIP layers is plotted in Fig.~\ref{fig:nmiplayer2d5} for an incident 2.5~GeV total energy neutron assuming a system of 15, 20, or 25 layers; or in other words, counting only MIP layers in the first 15, 20, or 25 HCAL layers (left to right). 
In Fig.~\ref{fig:nmiplayer2d5}, we can count the number of events in the 0 bin of each plot to find the fraction of neutron events which would {\it not} be vetoed, which we call {\it mis-vetoed} neutrons.
For an incident 2.5~GeV total energy neutron, the fraction of mis-vetoed depends strongly on the number of HCAL layers in the system.  
This suggests, for higher energy neutrons, that the mis-veto fraction is dominated by neutrons which do not interact in the system and are not contained.  

\begin{figure}[hbtp]
\begin{center}
    \includegraphics[width=0.3\textwidth]{images/hcal/nMIPLayers15_e2d5.pdf}
    \includegraphics[width=0.3\textwidth]{images/hcal/nMIPLayers20_e2d5.pdf}
    \includegraphics[width=0.3\textwidth]{images/hcal/nMIPLayers25_e2d5.pdf}    
    \caption{Number of MIP layers for a 2.5~GeV neutron produced with an incident angle of 0.0$^{\circ}$ at the front face of the calorimeter for a system of 15, 20, or 25 layers (left to right). }
 \label{fig:nmiplayer2d5}
 \end{center}
\end{figure}

The number of MIP layers is plotted in Fig.~\ref{fig:nmiplayer1d3} for an incident 1.3~GeV total energy neutron assuming a system of 15, 20, or 25 layers (left to right). 
Here, we note that the mis-veto efficiency is fairly constant as a function of the number of HCAL layers in the system.
We conclude that most mis-vetoed neutron events for lower energy neutrons do not deposit enough energy in the scintillator to pass the threshold for a MIP layer or are directly absorbed into the Steel absorber layer.

\begin{figure}[hbtp]
\begin{center}
    \includegraphics[width=0.3\textwidth]{images/hcal/nMIPLayers15_e1d3.pdf}
    \includegraphics[width=0.3\textwidth]{images/hcal/nMIPLayers20_e1d3.pdf}
    \includegraphics[width=0.3\textwidth]{images/hcal/nMIPLayers25_e1d3.pdf}    
    \caption{Number of MIP layers for a 2.5~GeV neutron produced with an incident angle of 0.0$^{\circ}$ at the front face of the calorimeter for a system of 15, 20, or 25 layers (left to right). }
 \label{fig:nmiplayer1d3}
 \end{center}
\end{figure}

Finally we conclude by benchmarking performance as a function of number of layers in the system and incident neutron angle.
This is shown in Fig.~\ref{fig:effmap1}.
Here we show the neutron mis-veto rate as a function of total energy, incident angle. and number of HCAL layers.  
First we see that the mis-veto rate fall quickly with energy which indicates that we are much more efficient at detecting higher energy neutrons at a mis-veto rate of $10^-3$-$10^-5$ depending on the number of layers in the system and the incident angle of the neutron.  
At higher energies, a larger incident angle or more layers reduces the mis-veto rate where in both cases the neutron traverses more material (absorber).
This indicates that the main source of mis-vetoed neutrons comes neutrons that do not interact with the HCAL.  
For lower energies, the mis-veto rate for neutrons is practically independent of incident angle and number of layers and is typically at a value of $10^-1$.  

\begin{figure}[hbtp]
\begin{center}
    \includegraphics[width=0.6\textwidth]{images/hcal/effmap1.pdf}
    \caption{Number of MIP layers for a 2.5~GeV neutron produced with an incident angle of 0.0$^{\circ}$ at the front face of the calorimeter for a system of 15, 20, or 25 layers (left to right). }
 \label{fig:effmap1}
 \end{center}
\end{figure}

In Fig.~\ref{fig:effmap1}, we show a more fine-grained mis-veto rate plot with more neutron energies considered.

\begin{figure}[hbtp]
\begin{center}
    \includegraphics[width=0.6\textwidth]{images/hcal/effmap2.pdf}
    \caption{Number of MIP layers for a 2.5~GeV neutron produced with an incident angle of 0.0$^{\circ}$ at the front face of the calorimeter for a system of 15, 20, or 25 layers (left to right). }
 \label{fig:nmiplayer1d3}
 \end{center}
\end{figure}

%A good starting point would be to focus on the event types that the ECal will certainly have a tough time with. These are the few-body photo-nuclear reactions for sure. So it might make sense to start with the ``bottom up'' study outlined above. 

{\color{red} to add: studies changing absorber thickness and varying MIP layer threshold. Further studies are on-going to better improve performance at lower energies. }

From single neutron studies, we conclude that we are able to veto neutrons at an event per $10^1 - 10^5$ for neutron kinetic energies ranging from 400~MeV to 2.5~GeV 
depending on number of layers in the system and incident angle.  
In the next Section, we study in more detail the veto capabilities of the entire calorimeter system, ECAL + HCAL, for rare photo-nuclear processes.





\subsection{Trigger}

We need to spell out our trigger streams. 

Signal stream: 

- a single cluster with energy below $\sim ??$ GeV. 

- coincident with a signal from the fast-or layer in the tracker

- rate? 

What other streams do we need to measure performance? 




%%%%%%%%%%%%%%
%   Budget   %
%%%%%%%%%%%%%%
\clearpage
\section{Budget and Schedule}

\subsection{DASEL}

\subsection{Tracking}

\subsection{Forward ECal}

\subsection{Hadronic Veto}

\subsection{Trigger}

\subsection{DAQ}

\subsection{Operations}



%%%%%%%%%%%%%%%%%%%%
%   Bibliography   %
%%%%%%%%%%%%%%%%%%%%
\bibliographystyle{JHEP}
\bibliography{bibliography}

\end{document}

