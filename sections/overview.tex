\section{Overview and Executive Summary (Editors: Philip Schuster)}

The LDMX experiment proposes a high-statistics search for low-mass dark matter at the DASEL beamline using the missing momentum technique, scattering incoming electrons in a tungsten target to produce dark matter via ``dark bremsstrahlung''. This clear signature is established by individually tagging incoming beam-energy electrons and unambiguously associating them with low energy, moderate transverse-momentum recoils and establishing the absence of a forward-going photon. The primary backgrounds are traditional bremsstrahlung processes with photo-nuclear reactions occurring in the target or forward calorimeter. Therefore, the experiment requires a high-speed, granular calorimeter with MIP sensitivity to identify rare photo nuclear reactions, in addition to low mass tracking that provides high-purity tagging for incoming electrons and clean, efficient reconstruction of recoils. The LDMX concept proposes to meet these challenges by leveraging technology under development for the HL-LHC and experience from the HPS experiment