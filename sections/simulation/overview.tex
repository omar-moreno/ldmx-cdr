
%\subsection{Simulations Overview (Editors: Natalia Toro, Jeremy McCormick)}
\subsection*{\draft{Simulations Overview (Editors: Natalia Toro, Jeremy McCormick)}}
This section describes the methods used to simulate signal and background physics reactions and the responses of various detectors to these reactions.  
For historical reasons, two separate simulation frameworks have been used: the ``tracker simulation'' implemented in SLIC comprises a full simulation of the tagger tracker and recoil tracker, with ECAL material included but not its detector response.  Likewise, the ``calorimeter simulation'' implemented directly in Geant4 comprises a full simulation of the ECAL and HCAL, with recoil tracker material included upstream. In both cases, the target material and magnetic field map are also included. \draft{Please check for accuracy} These two simulations are described in more detail in \S\ref{ssec:sim-tracker}--\ref{ssec:hcal}.

While the primary simulation engine is Geant4 \cite{g4}, the signal reaction (Dark Matter pair production) is modelled using an external generator based on MadGraph/MadEvent4 \cite{MG4}.  This generator, its validation, and the interface with Geant4 are described in \S\ref{ssec:signal}.  All background processes are modelled directly in Geant4, with modifications and biasing for photonuclear processes as discussed in \S\ref{ssec:photonuclear}.