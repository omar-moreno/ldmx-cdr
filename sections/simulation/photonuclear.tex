
\subsection{Photonuclear Model and Biasing (Editors: Natalia Toro, Omar Moreno)}
\label{ssec:photonuclear}
By default, Geant4 models photonuclear reactions initiated by $<3.5$ GeV incident photons using the Bertini Cascade model, and those initiated by $>3$ GeV photons using a high-energy string model (with a mixture of the two event generators in the intermediate region).  The Bertini model is, however, believed to be accurate up to $\sim 10$ GeV incident energies \cite{DennisPrivateCommunication}, and indeed is used in Geant4 for this purpose for other primary particles.  There are notable differences between the two models, particular for photon collisions with heavy nuclei, where the Bertini cascade within the nucleus frequently results in a large number of soft hadrons that are not well modeled  by the high-energy models.  For this reason, all LDMX simulations have been done using a modified FTFPBERT physics list, where photonuclear processes with $<10$ GeV photons are always modeled using the Bertini cascade.

The Bertini model is notably complete, including cross-sections from data up to almost 3 GeV for a large number of photon-proton interaction processes and for $2\rightarrow N$ and $3\rightarrow N$ processes within the nucleus (however, the final-state phase space is only taken  from data for $2\rightarrow 2$ processes, and process cross-sections above 3 GeV are extrapolated).  The critical photon energy range for LDMX is 3--4 GeV, and in some cases the phase space of multi-particle final states may be peaked and therefore poorly approximated by Geant.  Efforts are underway to validate the Bertini model at these energies against data for high-$Z$ nuclei. 