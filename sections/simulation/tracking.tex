
\subsection{Simulation of the Tagger and Recoil Trackers (Editors: Omar Moreno, Jeremy McCormick)}

The simulation of the passage of both charged and neutral particles through 
the tagger and recoil trackers as well as the target uses a software 
package, the Simulator for the Linear Collider (SLIC)~\cite{}, based on the
GEANT4 toolkit. It creates realistic charge depositions in the silicon layers,
simulates the APV25 readout chip amplifier chain, forms clusters and performs
track finding and fitting.  The simulation also contains the Ecal material,
needed to understand the contribution of backscattered particles to the 
occupancy of the last recoil tracker layer.
The simulation is used to understand track reconstruction efficiencies,
purity and to evaluate the momentum and vertex resolutions. A rendering of
the tagger and recoil tracker as used in the simulation is shown in
Figure~\ref{fig:ldmx_tracker_geo}.
%%%%%%%%%%%%%%
%   Figure   %
%%%%%%%%%%%%%%
\begin{figure}
    \centering
    \includegraphics[width=\textwidth]{images/simulation/wired_view_ldmx_tracker.png}
    \caption{Rendering of the tagger and recoil tracker geometry used for 
        performance studies.  The figure shows a typical event where an 
        incident electron (in blue) traverses through the tagger and recoil 
        trackers. The white boxes represent the tracker silicon microstrip
        sensors while the green lines represent the strips that were hit by
        the incident electron.}
    \label{fig:ldmx_tracker_geo}
\end{figure}

\subsubsection{Hit Reconstruction}

The energy depositions resulting from a particle traversing through the silicon
planes are used in the simulation of all physical processes resulting in 
current signals on the readout strips.  The signal present on each of the 
strips is used to simulate 
the APV25 pulse shape which is then sampled into a readout pipeline. 
Six of the samples from each strip, selected in such a way that it captures
the signal peak, are fit using a 3-pole function in order to extract the
rise and fall time, the amplitude and time of the hit, $t_0$.  The information
extracted from the fits is used to form strip hits which, in turn, are clustered
using a nearest neighbor algorithm.
%\begin{equation}
%    f(t) = A\frac{\tau_1^2}{(\tau_1 - \tau_2)^3}\left( e^{-\frac{t-t_{0}}{\tau_1}}
%        - \sum_{k=0}^2 \left(\frac{\tau_1 - \tau_2}{\tau_1\tau_2}(t-t_{0})\right)^k
%        \frac{e^{-\frac{t-t_{0}}{\tau_2}}}{k!} \right)
%\end{equation}
%where $\tau_1$ and $\tau_2$ represent the fall and rise time of the shaper 
%signal respectively.  The amplitude, $A$, and the time of the hit, $t_0$, are then
%determined from the fit.
%as follows: 
%\begin{itemize}
%    \item A list of seeds is created from all raw hits that have an amplitude, $S$,
%          $> 4\times \sigma_{\text{Noise}}$
%  \item Recursively add neighboring strips that have $S> 3 \times \sigma_{\text{Noise}}$
%          until a strip with $S < 3\times \sigma_{\text{Noise}}$ is found.
%      \item Require that neighboring hits have a $t_{0}$ that is within 8 ns of the seed hit.
%    \item Repeat the first two steps until seed strips are no longer found.
%\item Require that a cluster has an amplitude $> 4 \times \sigma_{\text{Noise}}$.
%\end{itemize}
%the cluster time is computed as the amplitude-weighted average
%of the $t_0$ times from the hits that compose it. 
Finally, all clusters in adjacent pairs
of layers are combined to create 3D hits used by the track finding and fitting
algorithm.

\subsubsection{Track Reconstruction}

The track finding and fitting algorithm proceeds in steps following a specific 
``tracking strategy''.  The strategies outline which layers are used and their
roles (e.g. seed, confirm, extend), 
the minimum number of hits required to form a track, kinematic constraints
(e.g. impact paremeters) and the maximum allowable $\chi^2$ value. The algorithm
proceeds as follows:
\begin{itemize}
    \item To begin, all possible combinations of hits on the specified seed 
          layers are used to form 3-hit track candidates.  A helix fit is then 
          perfomed on each of the candidates and only those which satisfy the
          $\chi^2$ requirement are kept.
    \item Once all seeds have been found, hits from a specified ``confirm'' 
          layer are added to the seeds and a helix fit is performed once again.
          Those 4-hit seeds which fail to satisfy the $\chi^2$ cut are eliminated.
    \item Next, tracks are ``extended'' using hits from the rest of the layers.
          After the addition of a hit, a helix fit is performed.  If the track
          fails the $\chi^2$ constraint, the hit is discarded.  This procedure
          is repeated until all hits in all extend layer have been added to all seeds. 
          However, it is possible for all hits in a layer to be discarded.
    \item Finally, all track candidates are merged in order to form a set of unique 
          tracks.  Tracks are allowed to share at most one hit.  If a pair of
          tracks shares more than a single hit, the track with the most hits is
          chosen.  If the two tracks have the same number of hits, the track 
          with the best $\chi^2$  is selected.
\end{itemize}

The performance of the track finding and fitting algorithm is discussed in 
section~\ref{}.  The tracking strategies used in these studies required only 
a single confirmation hit.  In the tagging (recoil) tracker, tracks were 
required to have at least 5 (4) hits. Furthermore, extremely loose cuts on the
track $\chi^2$ and distance of closes approach to the beam were imposed. 
Unless otherwise noted, the performance studies assumed that the 
tagger (recoil) tracker is within a uniform dipole field of strength -1.5 T 
(-.75 T), while all acceptance studies used the full field map.
