
\subsection{Simulation of the Tagger and Recoil Trackers (Editors: Omar Moreno, Jeremy McCormick)}

The simulation of the passage of both charged and neutral particles through 
both the tagger and recoil trackers uses a software package based on SLAC's 
org.lcsim infrastructure which wraps Geant4.  The simulation creates realistic
charge depositions in the silicon layers, simulates the APV25 readout chip 
amplifier chain as well as the 24 ns sampling of the signal, creates clusters 
and performs track finding.  The performance studies assumed that the 
tagger (recoil) tracker is within a uniform dipole field of strength -1.5 T 
(-.75 T), while all acceptance studies use the full field map.  A rendering of
the tagger and recoil tracker as used in the simulation is shown on Figure~\ref{}.

\subsubsection{Readout simulation}


\subsection{Hit Reconstruction}

The six samples emerging from each channel are fit using the following 3-pole 
function
\begin{equation}
    f(t) = A\frac{\tau_1^2}{(\tau_1 - \tau_2)^3}\left( e^{-\frac{t-t_{0}}{\tau_1}}
        - \sum_{k=0}^2 \left(\frac{\tau_1 - \tau_2}{\tau_1\tau_2}(t-t_{0})\right)^k
        \frac{e^{-\frac{t-t_{0}}{\tau_2}}}{k!} \right)
\end{equation}
where $\tau_1$ and $\tau_2$ represent the fall and rise time of the shaper 
signal respectively.  The amplitude, $A$, and the time of the hit, $t_0$, are then
determined from the fit.

Hits on neighboring strips are clustered using a nearest neighbor
algorithm as follows: 
\begin{itemize}
    \item A list of seeds is created from all raw hits that have an amplitude, $S$,
          $> 4\times \sigma_{\text{Noise}}$
  \item Recursively add neighboring strips that have $S> 3 \times \sigma_{\text{Noise}}$
          until a strip with $S < 3\times \sigma_{\text{Noise}}$ is found.
      \item Require that neighboring hits have a $t_{0}$ that is within 8 ns of the seed hit.
    \item Repeat the first two steps until seed strips are no longer found.
\item Require that a cluster has an amplitude $> 4 \times \sigma_{\text{Noise}}$.
\end{itemize}
After hits on a sensor have been
clustered, the cluster time is computed as the amplitude-weighted average
of the $t_0$ times from the hits that compose it. All clusters in adjacent pairs
of layers are combined to create 3D hits.

\subsubsection{Track Reconstruction}

The track finding and fitting algorithm proceeds in steps following a specific 
``tracking strategy''.  The strategies outline which layers are used, the 
minimum number of hits required to form a track and kinematic constraints.  
The algorithm proceeds as follows:
\begin{itemize}
    \item All possible combinations of hits are formed from the seed layers and
          and a helix fit is performed.  Only those seeds which satisfy a 
          $\chi^2$ requirement are kept. 
    \item Once all seeds have been found, hits from a specified ``confirm'' 
          layer are added to the seeds and a helix fit is performed once again.
          Those fits which do satisfy a $\chi^2$ cut are eliminated.
    \item Finally, tracks are ``extended'' by hits from the rest of the layers.
          After the addition of a hit, a helix fit is performed.  If the track
          fails the $\chi^2$ constraint, the hit is discarded.  This procedure
          is repeated until all hits in a layer have been added to all seeds, 
          however, it is possible for all hits in a layer to be discarded.
    \item All track candidates are merged in order to form a set of unique 
          tracks.  Tracks are allowed to share at most one hit.  If a pair of
          tracks shares more than a single hit, the track with the best $chi^2$ 
          is chosen. 
\end{itemize}

\textbf{Need to talk about what strategy and cuts are being used}.

