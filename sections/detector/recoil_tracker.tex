
\subsection{Recoil Tracker (Editors: Tim Nelson, Omar Moreno)}

The recoil tracker is designed to identify low-momentum (50 MeV - 1.2 GeV) recoils and precisely determine their momentum, direction and impact position at the target.  In addition, it must work together with the calorimeters to correctly distinguish low-momentum signal recoils from scattered beam electrons and multi-particle backgrounds. The key elements of the design are determined by this goal.  First, the recoil tracker is placed at the end of the magnet in the beginning of the fringe field to optimize tracking for particles up to two orders of magnitude softer than the beam energy electrons measured by the tagging tracker.  Second, the recoil tracker is short and wide for good acceptance in angle and momentum and to minimize the distance from the target to the calorimeters to improve their angular coverage. Finally, the recoil tracker provides 3-d tracking near the target for both direction and impact parameter resolution but emphasizes low mass density over the longest possible lever arm further downstream to deliver the best possible momentum resolution.  This design delivers good momentum resolution for both multiple-scattering limited low-momentum tracks and beam energy electrons that are nearly straight in the fringe field.

The layout of the recoil tracker is summarized in Table XXX and consists of four stereo layers immediately downstream of the target and two axial layers at larger intervals in front the the ECal.  The stereo layers are double-sided modules of silicon microstrips arranged at 15 mm intervals downstream of the target, with the first module centered at z=+7.5 mm.  These modules are laterally centered on the target and the center of the vacuum chamber and are identical to the modules of tagger tracker that are mounted upstream on the same support plate.  

The thinner axial-only layers, at z=+90 mm and z=+180 mm, are mounted on a separate support structure at the downstream end of the vacuum chamber as described in Section XXX and have a somewhat different module design.  Each module consists of a pair of sensors, glued end-to-end, with an APV25-based FR4 hybrid circuit board at each end of this structure to read out the two sensors. The sensors are standard p$^{+}$-in-n silicon microstrip sensors, but are somewhat shorter and wider than those used in the stereo modules and therefore require six APV25 chips to read out each sensor instead of five. The modules are supported at both ends by screw attachment of the hybrids to castellated support blocks attached to the cooled support structure. A vacuum compatible thermal compound is used to ensure good thermal conductivity between the hybrid and the support structure.