\subsection{Tracker Overview and Target (Editors: Tim Nelson, Omar Moreno)}

Although the tagging and recoil trackers function as two distinct systems, they use common technologies and share the same support structures and data acquisition hardware.  In particular, the first four layers of the recoil tracker are identical to the layers of the tagging tracker and share the same support and cooling structure.  The key element of this upstream support structure is a vertically-oriented aluminum plate onto which the stereo modules are mounted.  To provide cooling, a copper tube is pressed into a machined groove in the plate through which coolant flows.  This support plate slides from the upstream end of the vacuum chamber into precision kinematic mounts in a support box that is aligned to and locked in place inside the vacuum chamber.  Another similar plate slides into the support frame on the positron side of the chamber which hosts the Front End Boards (FEBs) that distribute power and control signals from the DAQ and digitize raw data from the modules for transfer to the external DAQ.  The last two layers of the recoil tracker, being much larger, are supported on another structure; a cooled support ring onto which the single-sided, axial-only modules are mounted. This support ring is installed from the downstream end of the chamber, engaging precision kinematic mounts in the support box for precise alignment to the upstream stereo modules. The cooling lines for all three cooled structures; the upstream and downstream tracker supports and the FEB support; are routed to a cooling manifold at the upstream end of the vacuum chamber which, in turn, connects to a cooling feedthrough with dielectric breaks on the bottom of the vacuum box.

It is the interposition of the target between the last layer of the tagging tracker and the first layer of the recoil tracker that clearly distinguishes between the target and recoil tracking systems. The target is a 350 micron tungsten sheet, comprising 10\% of a radiation length. This choice of thickness provides a good balance between signal rate and transverse momentum transfer due to multiple scattering which limits the utility of using transverse momentum as a signal discriminator. The tungsten sheet is glued to a 1 mm sheet of PVT scintillator which provides a fast level-0 signal to the ECal trigger indicating the passage of a beam electron in a given beam pulse.  The scintillator-tungsten sandwich is mounted in an aluminum frame which can be inserted into the upstream tracker support plate from the positron side to simplify the process of replacing or swapping the target.  With the scintillator mounted on the downstream face of the target, it can also be used to help identify events where a bremsstrahlung photon has undergone a photonuclear reaction in the target.

