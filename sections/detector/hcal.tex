\subsection{Hadronic Veto System (Editors: Jeremy Mans, Nhan Tran, Andrew Whitbeck)}

It is also important to extend the calorimeter to veto neutral hadrons being produced in photo-nuclear reactions.  
As exclusive reactions of this type are rare, the rates and radiation doses will be much lower as the design for the HG calorimeter is designed to fully contain the electromagnetic showers.  
This removes the requirement for silicon in the hadronic calorimeter section.  
A scintillator-based sampling calorimeter is a natural solution in this situation.  
The goal of the hadronic veto system is to identify neutral hadrons in the energy range from above approximately 100 MeV to several GeV with high efficiency.  
This requires a hadronic calorimeter with at least 5 nuclear interactions of depth in order to fully contain the most energetic of neutrons with greater than 99\% probability.  
Simultaneously, in order to detect lower energy neutrons, absorbing layers cannot be too thick such that neutrons of hundreds of MeV are captured in the absorbers.  
Therefore, a steel-scintillator (polystyrene) calorimeter of approximately 15 layers and totaling 5 nuclear interaction lengths is proposed.  
Each layer is structured as 50~mm of Steel, 2~mm air gap, 6~mm of scintillator, and 2~mm of air gap where the air gaps are left for detector services.  
The transverse size of each layer is 1~x~1~m to cover the solid angle of the signal acceptance.
Transverse granularity of the system is not required due to the lower rates expected in the hadronic system and in order to maintain high efficiency for neutral hadron detection.
An illustration of the hadronic veto system is given in Fig.~\ref{fig:hcal}.

\begin{figure}[hbtp]
\begin{center}
    \caption{Placeholder...
    {\color{blue} Drawing of HCAL, is it needed?  Could be integrated into a full detector rendering}
  }
 \label{fig:hcal}
 \end{center}
\end{figure}

Fast readout is also required of the hadronic calorimeter system in order to coincide with electromagnetic calorimeter information. 
Readout will be based on the CMS Phase 1 upgrade HCAL system which has fast readout capabilities intrinsically at frequencies of 40 MHz and would be sufficient for the DASEL beam structure.  
Scintillating light is read out by wavelength shifting fibers with silicon photomultipliers (SiPMs) as the photodetectors.
SiPM technology is chosen due to high gain and low noise capabilities.  
Each layer is read out by N wavelength shifting fibers and the light is optically summed into M photodetectors
In addition to the SiPMs, readout modules (48/64 channels per) and charge integration electronics designed for the CMS detector can be re-purposed for the LDMX experiment with minimal changes and takes advantage of experience of CMS collaborators on LDMX.  
The timescale for commissioning of the readout electrons in CMS are 2017/2018 and thus are appropriate for the LDMX timeline.

