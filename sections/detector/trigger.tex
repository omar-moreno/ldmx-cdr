
\subsection{Trigger System (Editors: Jeremy Mans, Nhan Tran, Andrew Whitbeck)}

The LDMX trigger system is designed to reduce the typical beam
particle arrival rate of $\approx 40$~MHz to a rate of \draft{O(1~kHz)} for
storage and analysis.  The selection is performed by a combination of
dedicated hardware and software running on general-purpose computers.

The first stage trigger is implemented in hardware and allows the
selection of both candidate events for dark matter production and
important samples for calibration and detector performance monitoring.
The overall trigger management is provided by a trigger manager board,
which receives inputs from the various triggering subsystems including
the silicon calorimeter and the scintillator calorimeter.  The latency
requirements on the trigger calculation latency are set \draft{by the tracker
  readout ASIC to 2us}.

The primary physics trigger is based on the silicon calorimeter and is
designed to select events with energy deposition significantly lower
than the full beam energy.  The silicon calorimeter ASIC calculates
the total energy in 2x2 fundamental cells for every 40~MHz
pseudo-bucket.  The energy information is transferred by digital data
link to the peripherary of the calorimeter, where sums can be made
over larger regions and transferred by optical link to the trigger
electronics.  The total energy is then used to select the events.

The use of a calorimeter trigger requirement of energy below a
threshold also requires a beam-presence measurement to avoid very high
trigger rates during crossings where there is no beam electron
present.  The trigger fiber hodoscope is designed to serve this
purpose.  The hodoscope is constructed of an array of \draft{1~mm}-diameter
scintillating fibers, which are mounted immediately upstream of the
tungsten target. \draft{[more details on the mechanics, including orientation
of the fibers]}

\begin{figure}[t]
  \caption{Drawing of the trigger fiber hodoscope}
\end{figure}

The fibers from the hodoscope are brought to an optical vacuum
feedthrough and a clear fiber ribbon is connected on the outside of
feedthrough to carry the light signals to the readout electronics.
Based on studies by the LHCb collaboration\cite{lhcb:scintfiberrad},
scaled to the diameter of fibers in use for the hodoscope, we expect a
typical signal of \draft{[]} photons to be produced in the hodoscope by a beam
electron after a total absorbed radiation dose of \draft{XXX Gray}.
We expect to able to bring at least \draft{[50\%]} of these photons
to the readout SiPMs.

The same electronics design is used for readout of the hodoscope and
the scintillator calorimeter.  This readout is based on SiPMs and the
housing is designed to allow operation of the SiPMs at reduced
temperatures (below $5^\circ$C) to reduce the thermally-induced
single-photon noise.  The typical photodetector efficiency of these
SiPMs is \draft{[30\%]}\cite{cms:sipm_performance}, providing a typical signal
of \draft{[]} PEs in the electronics.  The readout electronics will
continuously digitize the SiPM signal, providing an integrated charge
as well as time-of-arrival measurement for the pulse with an LSB of
500~ps.  Both amplitude and timing information can be provided to the
trigger, allowing the correction of the calorimeter amplitude for
timewalk effects already at trigger level.


