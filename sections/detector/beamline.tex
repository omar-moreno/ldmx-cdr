\subsection{Beamline (Editors: Tim Nelson, Omar Moreno)}

The LDMX beamline is relatively simple, consisting only of an analyzing magnet
and a vacuum chamber into which the tagging and recoil trackers are installed.
The analyzing magnet is a common 18D36 dipole magnet with a 14-inch vertical 
gap and operated at a central field of 1.5 T.  A stainless-steel vacuum
chamber with 1/2" walls fits just inside the magnet bore supporting the
tracking detectors and their readout electronics. The upstream end of the
vacuum chamber is closed by a plate with a 6" conflat flange for connection to
the upstream beampipe and the downstream end is closed by a titanium vacuum
window in front of the ECal face which sits 20 cm downstream of the target
at roughly the same $z$ position as the outer face of the magnet coils.
The magnet is rotated by approximately 100 mrad about the vertical axis with
respect to the upstream beamline so that the incoming 4 GeV beam follows the
desired trajectory to the target, with the incoming beam arriving at normal
incidence to, and centered on, the target which is laterally centered in the vacuum chamber at z=40 cm relative to the center of the magnet.  Although the specific dimensions
differ, this arrangement (aside from the vacuum window) is very similar
to the magnet and vacuum chamber employed by the HPS experiment at JLab.
As with HPS, additional vacuum boxes at the upstream end will be appended to the vacuum chamber as needed to accommodate feedthroughs for power, readout and cooling.

A number of 18D36 magnets, not currently in use, are in hand at SLAC along with the steel required to adjust the magnet gap, if required to suit our purposes. These include a magnet that is already assembled with a 14-inch gap as planned for LDMX.  This magnet was tested to 1.0 T in 1978 at which the power dissipated was 199 kW.  Based on the current capacity of the other similar magnets with smaller gaps, it is expected that this magnet can be operated at 1.5 T, resulting in a power dissipation of approximately 450 kW and requiring approximately 55 gpm flow of cooling water. If this magnet proves to be suitable for LDMX, it will simply have to be split, cleaned up, reassembled before testing and carefully mapping the field in the tracking volumes.

Since End Station A (ESA) is relatively distant from the critical areas in the Beam Switchyard, the vacuum requirements in ESA are quite modest: roughly $10^-4$ Torr or better.  Therefore, standard vacuum fabrication techniques for the vacuum chamber and vacuum window will clearly suffice here.  Furthermore, since HPS has achieved better than $10^-5$ Torr at JLab, it is clear that the construction techniques developed there will work here as well. However, as with HPS, a system of monitoring and interlocks capable of closing automatically controlled vacuum valves will be required to ensure that the liquid cooling system of the trackers cannot spoil the beam vacuum.

Although the final location of LDMX has not been determined, the upstream end of the ESA beamline has the most room and best access for locating the relatively large apparatus.  In particular, there is a suitable concrete pedestal for mounting the dipole with sufficient room on all sides to support the calorimeters and accommodate the required beamline elements.  

%The LDMX Detector is designed to be installed in End Station A at SLAC, but is
%relatively compact and could be installed at any of the locations being 
%considered for the experimental area of the DASEL beamline.

