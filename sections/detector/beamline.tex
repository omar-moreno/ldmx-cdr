
\subsection{Beamline (Editors: Tim Nelson, Omar Moreno)}

The LDMX beamline is relatively simple, consisting only of an analyzing magnet and a vacuum chamber into which the tagging and recoil trackers are installed.  The analyzing magnet is a common 18D36 dipole magnet with a 12-inch vertical gap and operated at a central field of 1.5 T.  A stainless-steel vacuum chamber with 1/2" walls fits just inside the magnet bore supporting the tracking detectors and their readout electronics. The upstream end of the vacuum chamber is closed by a plate with a 6" conflat flange for connection to the upstream beampipe and the downstream end is closed by a titanium vacuum window in front of the ECal face which sits 20 cm downstream of the target at roughly the same $z$ position as the outer face of the magnet coils.  The magnet is rotated by approximately 100 mrad about the vertical axis with respect to the upstream beamline so that the incoming 4 GeV beam follows the desired trajectory to the target, with the incoming beam arriving at normal incidence to, and centered on, the target.  Although the specific dimensions differ, this arrangement (aside from the vacuum window) is very similar to the magnet and vacuum chamber employed by the HPS experiment at JLab.  As with HPS, additional vacuum boxes at the upstream and/or downstream ends will be appended to the vacuum chamber to accommodate feedthroughs for power, readout and cooling.  


%The LDMX Detector is designed to be installed in End Station A at SLAC, but is relatively compact and could be installed at any of the locations being considered for the experimental area of the DASEL beamline.