\subsection{Tagging Tracker (Editors: Tim Nelson, Omar Moreno)}

The tagger tracker is designed to unambiguously identify incoming beam electrons with the correct energy and precisely determine their momentum, direction, and impact position at the target.  The key elements of the design are determined by this goal.  First, the layout of the tagger tracker accepts only beam electrons that have the roughly the momentum and trajectory expected for the incoming beam to eliminate any off-energy beam electrons.  Second the layers have low mass and are spaced far apart in a significant magnetic field to ensure a precise estimate of momentum.  Finally, a large number of layers ensures the redundancy required to ensure high-purity pattern recognition.

The layout of the tagging tracker is summarized in Table XXX. The tagging tracker consists of six double-sided modules of silicon microstrips arranged at 10 cm intervals upstream of the target, with the first module centered at z=-7.5 mm.  The modules are positioned laterally in the vacuum chamber so that they are centered along the path of incoming 4 GeV beam electrons.

Each module places a pair of 4 cm $\times$ 10 cm sensors back to back; one sensor with vertically oriented strips for the best momentum resolution and the other at 100 mrad stereo angle to improve pattern recognition and provide three-dimensional tracking. The sensors are standard p$^+$-in-n type silicon microstrip sensors identical to those used for HPS.  These sensors have 30(60) $\mu$m sense(readout) pitch to provide excellent spatial resolution at high S/N ratios and are operable to at least 350V bias for radiation tolerance.

The sensors are read out with CMS APV25 ASICs operated in multi-peak mode, which allows for reconstruction of hit time with a resolution of approximately 2 ns. At very high occupancies, six sample readout can be used to distinguish hits that overlap in time down to 50 ns, which limits the readout rate, and therefore the trigger rate, to approximately 50 kHz.  However, at lower occupancies as anticipated at LDMX, three-sample readout may suffice, enabling a maximum trigger rate approaching 100 kHz.  These sensors are mounted on standard FR4 hybrid circuit boards which provide the power conditioning and I$^2$C control for the APV25, as well as a thermal path to the cooled support structure.

The sensors and hybrids are assembled into half-modules, which are the smallest non-rebuildable units of the tagger tracker.  Each half module consists of a single sensor, glued to the hybrid at one end with conductive epoxy to provide bias voltage and support.  At the other end of the sensor a thin ceramic plate is glued to support the other end of the sensor.  A pair of half modules is attached to either side of an aluminum module support with screws to form a module.  A vacuum compatible thermal compound between the half-module and the support ensures a good cooling path.  The double-sided modules are attached to large vertically-oriented aluminum plate to position them along the beamline.  A copper tube is pressed into a machined groove in this support and cooling plate to provide cooling to the system and the plate is, in turn, kinematically mounted inside of a support frame that fits inside the vacuum chamber.  This support arrangement places only silicon in the tracking volume and no material on the side of the tagger towards which electrons bend to eliminate secondaries that would be generated by degraded beam electrons hitting material close to the tracking planes.  

Overall, this design is similar to that of the HPS tracker with some important simplifications.  First, because the radiation dose here is modest, cooling is only needed to remove heat from the APV25 chips and not to keep the silicon itself cold.  Therefore, water cooling close to room temperature can be used and the significant issues of differential thermal expansion can be ignored.  Second, the detector here is in no danger from the nominal beam, so it does not need to be remotely movable as in HPS.  Finally, the vacuum chamber here is of a very generous size to achieve the desired acceptance in the ECal and recoil tracker, so the vacuum chamber is not crowded, as is the case with HPS.  In summary, the tagger tracker is nearly identical to, but is a significant simplification of, the tracking system built for the HPS experiment.  For this reason, the risk associated with this project is small.



