
\subsection{Tagging Tracker Performance (Editors: Tim Nelson, Omar Moreno)}


The tagging tracker must identify incoming beam-energy electrons with extremely high purity, suppressing the mis-reconstruction of any incoming low-momentum charged particles as beam energy electrons. In particular, any incoming charged particle within the recoil acceptance for signal that is reconstructed as having the beam energy in the tagging tracker is an irreducible background. The design of the tagging tracker makes the likelihood of such errors vanishingly small, with good resolution for both beam energy and off-energy incoming tracks and an exceedingly low rate of mis-reconstruction for tracks within the recoil energy acceptance.

In order for an incoming low momentum particle to fake a beam energy electron in the tagger, a number of conditions must simultaneously be met:
\begin{enumerate}
\item The incoming particle must reach the first tagger layer or it will not intersect with any material until it hits the wall of the vacuum chamber.
\item The particle must either scatter in each layer in order to fake a 4 GeV track or create secondaries that generate occupancy which confuses the pattern recognition in the tracker resulting in reconstruction of a fake 4 GeV track.
\item The resulting track must have a trajectory consistent with that of a typical 4 GeV beam electron all the way through the tracker.
\item The resulting track must have an impact point at the target consistent with a reconstructed track within the signal acceptance in the recoil tracker. 
\end{enumerate}
Using an analytic model of the tagging tracker that includes the effect of intrinsic resolutions and multiple scattering in the tracker planes, it is evident that each of these requirements places a very heavy penalty on any off-energy component in the beam.  First, incoming particles with less than approximately 500 MeV momentum will not hit the first layer of the tagger unless they are significantly off-trajectory as well.  Furthermore, even at 500 MeV, a first scatter of more than 10$^\circ$ is needed in order for the incoming particle to appear to be on the correct trajectory. It is clear then that the most challenging scenario is large contamination with incoming charged particles at the top of the momentum range for signal recoils, nominally 1.2 GeV.  Such particles have the highest likelihood of reaching the first layer of the tagger tracker without being bent away by the magnetic field and require much smaller scatters and/or track reconstruction errors to result in fake tags. In order for a 1.2 GeV particle to make a trajectory through the tracker consistent with a 4 GeV track, six successive scatters of approximately ten milliradians must occur, each equivalent to approximately 15$\sigma$ on the multiple scattering distribution. From the tails of the Moliere scattering distribution, the likelihood of each of these scatters is smaller than one per million.  Therefore, the much more likely scenario is the generation of secondaries an the material of the tagger tracker followed by misreconstruction of a fake 4 GeV track.  Since the resulting 4 GeV track must arrive at the target on the correct trajectory and beam energy electrons arrive normal to the target with a one-sigma error of 250 microradians, there is very little phase space for randomly reconstructed 4 GeV fakes to have the correct trajectory.  Finally, any falsely reconstructed 4 GeV track must have a common impact point in the target with a real track of matching momentum in the recoil tracker, which is unlikely for a falsely reconstructed tracks.

In order to more fully test these scenarios, two samples of incident electrons were simulated and reconstructed in the tagger tracker.  The first is a sample of XXX 4 GeV electrons on the nominal beam trajectory.  The second is a sample of XXX 1.2 GeV electrons on a trajectory that allows them to pass through all seven layers of the tagging tracker.  The 4 GeV sample confirms the expected resolutions, as shown in Figure~\ref{fig:trackin_4gev}.  
%=======================
\begin{figure}[htp]
\begin{center}
%    \includegraphics[width=7cm]{figures/dummy}
\caption{\small{Reconstructed total momentum, momentum transverse to target and x-y position at the target for a sample of 4 GeV beam electrons.} }
\label{fig:tracking_4gev}
\end{center}
%\vspace*{-5mm}
\end{figure}
%=======================
These indicate that tight requirements can be made in both the energy and trajectory at the target that rejects off-momentum particles that could be present in the incoming beam and that the tagging tracker identifies a precise impact position that can be used for tracking recoil candidates.  The 1.2 GeV sample confirms at the level of 1 part in 10$^?$ that these tracks cannot be mistaken for 4 GeV tracks, as shown in Figure~\ref{fig:tracking_1pt2gev}.  
%=======================
\begin{figure}[htp]
\begin{center}
%    \includegraphics[width=7cm]{figures/dummy}
\caption{\small{Reconstructed total momentum for a sample of 1.2 GeV beam electrons.} }
\label{fig:tracking_1pt2gev}
\end{center}
%\vspace*{-5mm}
\end{figure}
%=======================
In order to probe the likelihood of reconstructing fake 4 GeV tracks at higher statistics we further introduce random noise hits on all planes of the tracker at rates of $10^{-3}$, $10^{-2}$ and $10^{-1}$ in all planes of the tracker, where typical noise occupancies in similar HPS modules are roughy $10^{-4}$. Figure~\ref{fig:tracking_1pt2gev_noise} shows the resulting distribution of reconstructed tracks in energy and $p_T$ at the target. Obviously, such extreme occupancies in the tracker are atypical, and would easily be selected against with negligible impact on signal efficiency. 
%=======================
\begin{figure}[htp]
\begin{center}
%    \includegraphics[width=7cm]{figures/dummy}
\caption{\small{Reconstructed momentum vs. momentum transverse to the target for a sample of 1.2 GeV electrons in the presence of $10^{-3}$, $10^{-2}$ and $10^{-1}$ random occupancy in all sensors.} }
\label{fig:tracking_1pt2gev_noise}
\end{center}
%\vspace*{-5mm}
\end{figure}
%=======================
Furthermore, these fake tracks, not being due to an individual low-momentum track, will typically not align with a low-momentum track in the recoil tracker.  Although further study will be required to find the beam intensity limits for this tagging tracker design, we can safely conclude that it is more than capable of providing the tagging purity required for the first stage of the LDMX experiment.


\subsection{Recoil Tracker Performance, (Editors: Tim Nelson, Omar Moreno)}

Needs to be spelled out in more detail...
\begin{itemize}
    \item Acceptance
    \item Efficiency 
    \item What kind of activity cuts do we want to apply?  Background rejection
          power?  Signal efficiency? 
\end{itemize}




