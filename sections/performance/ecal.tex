
\subsection{Forward Electromagnetic Calorimeter (Editors: Joe Incandela)}

Owen and Joe will discuss this at the July 8 meeting. But Philip's notes include: 

- Hermiticity:  make sure to include a study of cracks or dead material in the detector simulation. Do we need to worry about this? Why? 

Large scale ``top down'' monte carlo study to demonstrate baseline performance of ECal and to justify more detailed study of specific reactions that dominate the tail of low energy deposition events. We need to quantify everything I'm about the say more carefully. Starting from $4\times 10^{14}$ EOT, the baseline tracker selections bring the event sample down to $\sim 4 \times 10^{12}$. So we're dealing with $\sim 4 \times 10^{12}$ events with a soft recoiling electron and a hard, $\sim 3$ GeV, photon. ECal events that are hadron rich occur about $\sim10^{-3}$ of the time. So now we're down to $\sim 4\times 10^{9}$ hadron rich events in the ECal. 

- most importantly, we want to understand what dominates the low energy deposition events

- we want to characterize the hadron rich events (because we know they are a potential issue)

- explain veto strategy

- at what point is the energy deposition so low that it's not possible to veto effectively? what are these events types? 

Specific ``bottom up'' studies of photo-nuclear reactions:  We know that certain event types could pose a challenge, so let's study them. The numbers shown below are with very loose kinematic selections, so they are upper bounds. They are also for $9$ GeV photons, so Philip and Natalia will need to correct them. This is a good starting point for study however: 

- $\gamma N \rightarrow (\rho,\omega,\phi)N\rightarrow \pi^+\pi^- N$ ($\lesssim 10^8$ of this event type). 

- $\gamma N \rightarrow  \mu^+\mu^- N$ ($\lesssim 2 \times 10^7$ of this event type). 

- $\gamma p \rightarrow \pi^+ n$ ($\lesssim 4\times 10^5$ with $\sim 4\times 10^3$ of these having a backscattered $\pi^+$). 

- $\gamma n \rightarrow n \bar{n} n$ ($\lesssim 4\times 10^5$ of this type). 

- $\gamma (p,n) \rightarrow K_L K_L + X$ (expect this at the $\sim 10^3$ level, but we need to check this!)

The current plan is to use a particle gun and weight the depth of origination and angle/energy distribution using data. Let Philip and Natalia know when you're ready to do this. 
{\it We need to be especially careful to include the regions of phase space where the MIPs are soft or wide/back scattering by recoiling off the nucleons or atoms. This needs a dedicated study, starting with the physics simulations group, P,N,E,G}. 


